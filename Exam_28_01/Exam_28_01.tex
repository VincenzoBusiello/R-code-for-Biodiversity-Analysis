\documentclass{beamer} % for presentations 
\usepackage{graphicx} % package for images
\usepackage{listings} % package for code lines
\usepackage{multicol} % package for creating multiple columns on each slide
\usepackage{url}

\definecolor{myred}{rgb}{0.6,0,0} % a color for comments on scripts slides


\usetheme{Dresden}
\usecolortheme{default}

\title{Quantifying aquatic vegetation with NDAVI}
\author{Vincenzo Busiello\\ 
\small\url{https://github.com/VincenzoBusiello}}
\date{28/01/2026}

\begin{document}

\maketitle

\AtBeginSection[]
{
\begin{frame}
\frametitle{Outline}    
    \tableofcontents[currentsection]
\end{frame}
}

\section{Bourgneuf Bay}

        \begin{frame}{Bourgneuf Bay}
            \begin{multicols}{2}
                \begin{itemize}
                    \item {\scriptsize French Atlantic Coast} 
                    \item {\scriptsize Department of Loire-Atlantique}
                    \item {\scriptsize Wetland zone protected by the Ramsar Convention}
                \end{itemize}
            \columnbreak
                \begin{center}
                    \includegraphics[width=0.40\textwidth]{bay_view.jpg}
                \end{center}
            \end{multicols}
        \end{frame}

        \begin{frame}{Why this site?}
            \begin{multicols}{2}                
                \begin{center}
                    \includegraphics[width=0.40\textwidth]{bourgneuf_sat.png}
                \end{center}
            \columnbreak
                \begin{itemize}
                    \item The presence of Zostera noltii and other photosynthetic aquatic organisms                    
                    \item Site protected by the Noirmoutier island
                    \item Already used for the same type of studies
                \end{itemize}
            \end{multicols}
        \end{frame}


\section{Codes and Data}


        \begin{frame}{Data}
            All the images used for this project were downloaded from the website \url{https://browser.dataspace.copernicus.eu} using the follow criteria:
                \begin{itemize}
                    \item An area was selected that covers a portion of the entire geographical site previously studied (Maria Laura Zoffoli et al.).
                    \item The time period is the same for all three years chosen: September.
                    \item Cloud cover less than 30 percent
                    \item Download of bands 2,3,4 and 8 in .tiff extension in 8 bit
                \end{itemize}
        \end{frame}

        \begin{frame}{Normalized Difference Aquatic Vegetation Index}
            \begin{equation}
                NDAVI = \frac{NIR - BLUE}{NIR + BLUE}
            \end{equation}
        \end{frame}
        
        \begin{frame}{Packages}
            \begin{itemize}
                \item    \texttt{library(terra)} 
                \item    \texttt{library(imageRy)} 
                \item    \texttt{library(viridis)}
                \item    \texttt{library(ggplot2)} 
                \item    \texttt{library(patchwork)}
            \end{itemize}
        \end{frame}
        
        \begin{frame}{Functions}
                \begin{multicols}{2}
                    \texttt{setwd()}\\
                    \texttt{list.files()}\\
                    \texttt{vector()}\\
                    \texttt{length()}\\
                    \texttt{seq\_along()}\\
                    \texttt{rast()} \\
                    \texttt{c()}\\
                    \texttt{par()}\\ 
                    \texttt{im.plotRGB()}\\
                    \texttt{dev.off()}\\
                    \texttt{plot()}\\
                    \texttt{im.classify()}\\
                \columnbreak
                    \texttt{ncell()}\\
                    \texttt{freq()}\\
                    \texttt{data.frame()}\\
                    \texttt{View()}\\
                    \texttt{ggplot()}\\
                    \texttt{aes()}\\
                    \texttt{geom\_bar()}\\
                    \texttt{scale\_color\_viridis\_d()}\\
                    \texttt{ylim()}\\
                    \texttt{xlab()}\\
                    \texttt{ylab()}\\
                \end{multicols}
        \end{frame}

        \begin{frame}{Creation of 4 bands Images}
            \lstinputlisting[language=R, firstline=46, lastline=53, commentstyle=\color{myred}, basicstyle=\footnotesize]{ScriptEsame.R}
        \end{frame}
        
        \begin{frame}{TrueColor vs. FalseColor - Code}
            \lstinputlisting[language=R, firstline=73, lastline=87, commentstyle=\color{myred}, basicstyle=\footnotesize]{ScriptEsame.R}
        \end{frame}

        \begin{frame}{TrueColor vs. FalseColor - Images}
            \begin{center}
                \includegraphics[width=0.55\textwidth]{falsecolorcomp.png}
            \end{center}
        \end{frame}


        \begin{frame}{NDAVI - Code}
            \lstinputlisting[language=R, firstline=91, lastline=104, commentstyle=\color{myred}, basicstyle=\footnotesize]{ScriptEsame.R}
        \end{frame}

        \begin{frame}{NDAVI - Images}
            \begin{center}
            \begin{multicols}{3}
                \includegraphics[width=0.3\textwidth]{FalseColor20.png}
                \includegraphics[width=0.3\textwidth]{FalseColor23.png}
                \includegraphics[width=0.3\textwidth]{FalseColor25.png}
            \end{multicols}
            \end{center}
        \end{frame}

        \begin{frame}{Classification - Code}
            \lstinputlisting[language=R, firstline=117, lastline=120, commentstyle=\color{myred}, basicstyle=\tiny]{ScriptEsame.R}
        \end{frame}


        \begin{frame}{Percentages of Covers for each Class}
            \begin{center}
                \includegraphics[width=0.5\textwidth]{Tabella.png}
            \end{center} 
        \end{frame}
        
        \begin{frame}{Surfaces Classification}
            \begin{center}
                \includegraphics[width=1\textwidth]{bars.png}
            \end{center}
        \end{frame}


\section{Conclusions}

        \begin{frame}{Conclusions}
                During 2020, the aquatic photosynthetic component was much more present 
                than in the other two years, like twice as much as the other years. This 
                could be explained by a possible difference in water level in the last two 
                years compared to the first, or by the presence of more turbid waters that 
                masked the marine vegetation component. \\
                Perhaps masking the terrestrial portion of the image, the part involving 
                the island of Noirmoitier, would have provided more consistent data. The
                use of other indices such as WAVI (Water-Adjusted Vegetation Index) might 
                be more appropriate in these cases, but it requires field studies to measure. 
        \end{frame}

        
\section{Websites}
        \begin{frame}{Websites}
            \begin{itemize}
                \item \url{https://browser.dataspace.copernicus.eu} \\
                \item \url{https://en.wikipedia.org/wiki/Bay_of_Bourgneuf} \\
                \item \url{https://www.earthdata.nasa.gov/learn/trainings/spectral-indices-land-aquatic-applications}\\
            \end{itemize}  
        \end{frame}


\end{document}
